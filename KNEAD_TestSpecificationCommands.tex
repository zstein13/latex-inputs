%%%%%%%%%%%%%%%%%%%%%%%%%%%%%%%%%%%%%%%%%%%%%%%%%%%%%%%%%%%%%
% beginning of Test Procedure Macro, and auxillary commands %
%%%%%%%%%%%%%%%%%%%%%%%%%%%%%%%%%%%%%%%%%%%%%%%%%%%%%%%%%%%%%

%%% 20210913WLC added this toggle (uses etoolbox package) to allow for not starting a test on a new page
\newtoggle{toggleTestProcStartOnNewPage}
\settoggle{toggleTestProcStartOnNewPage}{false}% it should be set in \newtoggle by default to false, but this shows how to set the variable by example
\settoggle{toggleTestProcStartOnNewPage}{true}%  it should be set in \newtoggle by default to false, but this shows how to set the variable by example

%WLC 20211017 added command so user an change pass/fail text if desired
\newcommand{\tpPassFailText}{PASS \hfil FAIL}
%\renewcommand{\tpPassFailText}{Verified}

%WLC 20211019 added command so user can change the test in the notes section
\newcommand{\tpResultsText}{Results~:}%~ used to keep spacing clean
%\renewcommand{\tpResultsText}{Notes/Initials~:}%~ used to keep spacing clean

%%%%%%%%%%%%%%%%%%%%%%%%%%%%%%%%%%%%%%%%%%%%%%%%%%%%%%%%%%%%%%%%%%%%%%%%%%%%%%%%%%%%%%%%%%%%%%%%%%%%%%%%%%%%%%%%%%%%%
% commands to define an \hline that won't allow a page break, ala \\* in longtable
% http://tex.stackexchange.com/questions/6350/how-to-disable-pagebreak-on-hline-in-longtable
%%%%%%%%%%%%%%%%%%%%%%%%%%%%%%%%%%%%%%%%%%%%%%%%%%%%%%%%%%%%%%%%%%%%%%%%%%%%%%%%%%%%%%%%%%%%%%%%%%%%%%%%%%%%%%%%%%%%% 
\makeatletter
\def\nobreakhline{%
\noalign{\ifnum0=`}\fi
 \penalty\@M
\futurelet\@let@token\LT@@nobreakhline}
\def\LT@@nobreakhline{%
\ifx\LT@next\hline
  \global\let\LT@next\@gobble
 \ifx\CT@drsc@\relax
   \gdef\CT@LT@sep{%
     \noalign{\penalty\@M\vskip\doublerulesep}}%
 \else
   \gdef\CT@LT@sep{%
     \multispan\LT@cols{%
       \CT@drsc@\leaders\hrule\@height\doublerulesep\hfill}\cr}%
 \fi
\else
 \global\let\LT@next\empty
 \gdef\CT@LT@sep{%
   \noalign{\penalty\@M\vskip-\arrayrulewidth}}%
\fi
\ifnum0=`{\fi}%
\multispan\LT@cols
 {\CT@arc@\leaders\hrule\@height\arrayrulewidth\hfill}\cr
\CT@LT@sep
\multispan\LT@cols
 {\CT@arc@\leaders\hrule\@height\arrayrulewidth\hfill}\cr
\noalign{\penalty\@M}%
\LT@next}
\makeatother
%%%%%%%%%%%%%%%%%%%%%%%%%%%%%%%%%%%%%%%%%%%%%%%%%%%%%%%%%%%%%%%%%%%%%%%%%%%%%%%%%%%%%%%%%%%%%%%%%%%%%%%%%%%%%%%%%%%%%
% end \nobreakhline
%%%%%%%%%%%%%%%%%%%%%%%%%%%%%%%%%%%%%%%%%%%%%%%%%%%%%%%%%%%%%%%%%%%%%%%%%%%%%%%%%%%%%%%%%%%%%%%%%%%%%%%%%%%%%%%%%%%%%

\newcounter{NTPRC}
\newlistof[NTPRC]{testprocedure}{aux_tproc}{\listtestproceduresname}
\newcounter{NTPS}
\newlistof[NTPS]{testprocstep}{aux_tstep}{\listtestprocstepsname}

\newcommand{\numberedtestprocstep}[2]{%
\refstepcounter{testprocstep}
\par\noindent\textbf{S-{#1}}
\addcontentsline{tss}{testprocstep}{\protect\numberline{#1}#2}\par}%use tss to avoid confusion with texnic .tps files


\newcommand{\numberedtestprocedure}[2]{%
\refstepcounter{testprocedure}
\addcontentsline{aux_tproc}{testprocedure}{\protect\numberline{#1}#2}\par}
%%%\par\noindent\textbf{\begin{center} Test Procedure {#1} \end{center} \newline \begin{center} {#2}\end{center}}


% length for total width of the test proc table, this subtracts off space for the column edge bars
\newlength{\TestProcWidth}%
%\setlength{\TestProcWidth}{6.35in}% adds in a smidge to account for the | at the ends
\setlength{\TestProcWidth}{\textwidth}
\addtolength{\TestProcWidth}{-0.15in}% subtracts a smidge to account for the | at the ends
%
\newlength{\TestProcRightWidth}%
%\setlength{\TestProcRightWidth}{4.93in}% emperical length determined from trial and error
%                               6.5 - 0.15 - 4.93 = 1.42
\setlength{\TestProcRightWidth}{\textwidth}% emperical length delta determined from trial and error
\addtolength{\TestProcRightWidth}{-1.6in}%wlc20210909 increased from 1.42 since right columns were running wide
\newlength{\TestProcIconOffset}%
\setlength{\TestProcIconOffset}{20pt}% emperical length determined from trial and error
\newlength{\TestProcIconWidth}
\setlength{\TestProcIconWidth}{0.5in}
\newlength{\TestProcFigureWidth}%
\setlength{\TestProcFigureWidth}{\TestProcRightWidth}% adds in a smidge to account for the | at the ends
%
\newcommand{\ThisTestProcStepLabel}{?!?ThisTestProcStepLabel?!?} %this gets set and reset in \tpStepLabeled[4]
%
\newcounter{TestProcItemCounter}
%
\newcommand{\ThisTestProcNum}{?!?ThisTestProcNum?!?} %this gets set and reset in \TestProc[6]
\newcommand{\ThisTestProcName}{?!?ThisTestProcName?!?} %this gets set and reset in \TestProc[6]
\newcommand{\ThisTestProcLabel}{?!?ThisTestProcLabel?!?} %this gets set and reset in \TestProc[6]

%
% command to provide single line TestProc steps with an ICON in the left side
% these are not numbered so they cannot be referenced :(
%
\newcommand\tpStepICONBASE[2]{
\begin{minipage}[c][][c]{\TestProcIconWidth}\includegraphics{#2}\end{minipage}
&\begin{minipage}[c][][c]{\TestProcRightWidth}\vspace{4pt}\bf#1\vspace{4pt}\end{minipage}\\*
\hline \hline%
}


%
\newcommand{\tpStepCLOCK}[1]{\tpStepICONBASE{#1}{/Users/zsteinberg/UMD/ENPM818I/LatexInputs/MarginIcons/CLOCK.pdf}}
%
\newcommand{\tpStepHAND}[1]{\tpStepICONBASE{#1}{/Users/zsteinberg/UMD/ENPM818I/LatexInputs/MarginIcons/HAND.pdf}}
%
\newcommand{\tpStepINFO}[1]{\tpStepICONBASE{#1}{/Users/zsteinberg/UMD/ENPM818I/LatexInputs/MarginIcons/INFO.pdf}}
%
\newcommand{\tpStepKEY}[1]{\tpStepICONBASE{#1}{/Users/zsteinberg/UMD/ENPM818I/LatexInputs/MarginIcons/KEY.pdf}}
%
\newcommand{\tpStepMAGNIFY}[1]{\tpStepICONBASE{#1}{/Users/zsteinberg/UMD/ENPM818I/LatexInputs/MarginIcons/MAGNIFY.pdf}}
%
\newcommand{\tpStepPLAYARROW}[1]{\tpStepICONBASE{#1}{/Users/zsteinberg/UMD/ENPM818I/LatexInputs/MarginIcons/PLAYARROW.pdf}}
%
\newcommand{\tpStepBANG}[1]{\tpStepICONBASE{#1}{/Users/zsteinberg/UMD/ENPM818I/LatexInputs/MarginIcons/BANG.pdf}}
%
\newcommand{\tpStepSHOCK}[1]{\tpStepICONBASE{#1}{/Users/zsteinberg/UMD/ENPM818I/LatexInputs/MarginIcons/SHOCK.pdf}}
%
\newcommand{\tpStepRADIATION}[1]{\tpStepICONBASE{#1}{/Users/zsteinberg/UMD/ENPM818I/LatexInputs/MarginIcons/RADIATION.pdf}}
%
\newcommand{\tpStepQUESTION}[1]{\tpStepICONBASE{#1}{/Users/zsteinberg/UMD/ENPM818I/LatexInputs/MarginIcons/QUESTION.pdf}}
%
\newcommand{\tpStepBULLSEYE}[1]{\tpStepICONBASE{#1}{/Users/zsteinberg/UMD/ENPM818I/LatexInputs/MarginIcons/BULLSEYE.pdf}}

%
% macro to include a figure in either the action or expected results part of the step
%
\newcommand{\tpStepFigure}[2]
{%
	\par
	{
	\ifthenelse{\lengthtest{#2>\TestProcFigureWidth}}{\setlength{\TestProcFigureWidth}{\TestProcRightWidth}}{\setlength{\TestProcFigureWidth}{#2}}
		\hfill\includegraphics[width=\TestProcFigureWidth]{#1}\hfill
	}
}
%
% the macros used within the \TestProcedure[6] macro to make things easier for test writers 
%
% \tpRqmt[1] - autonumbers requirements met by this test case
%
% #1 is the text of the requirement, usually something like SPS-1.2.3.4.5 or similar
%
\newcommand{\tpRqmt}[1]{\stepcounter{TestProcItemCounter}~R-\ThisTestProcNum-\theTestProcItemCounter & \parbox[top][][c]{\TestProcRightWidth}{\vspace{2pt}#1\vspace{2pt}} \\*
\hline%
}%

%
% base macro for a test step
%
\newcommand{\tpStepBase}[5]{\renewcommand{\thetestprocstep}{#5}
{\bf \numberedtestprocstep{#5}{#4}\label{#4}} & \parbox[top][][c]{\TestProcRightWidth}{\vspace{2pt}#1\vspace{2pt}} \\*
\nobreakhline%
Expected Result& \parbox[top][][c]{\TestProcRightWidth}{\vspace{2pt}#2\vspace{2pt}}\\*
\nobreakhline%
\tpPassFailText& \parbox[top][#3][t]{\TestProcRightWidth}{\vspace{2pt}\tpResultsText}\\*
\hline \hline%
}%


%
\newcommand{\tpStep}[3]{\stepcounter{TestProcItemCounter}~\tpStepBase{#1}{#2}{#3}{S-\ThisTestProcNum-\theTestProcItemCounter}{\ThisTestProcNum-\theTestProcItemCounter}}
\newcommand{\tpStepLabeled}[4]{\stepcounter{TestProcItemCounter}~\tpStepBase{#1}{#2}{#3}{#4}{\ThisTestProcNum-\theTestProcItemCounter}}
%
% \tpStepLevel[3] - autonumbered test warning for this test case
%
% #1 is procedure step level (Info, warning, danger, etc.) 
% #2 is Warning message
% #3 is height of comment box
%       usually some multiple of 12 pts, which is a normal line height, but not constrained to a multiple of 12 pt e.g. [42pt]
%
\newcommand{\tpStepLevel}[3]{\stepcounter{TestProcItemCounter}~{\bf S-\ThisTestProcNum-\theTestProcItemCounter} & \parbox[top][][c]{\TestProcRightWidth}{\vspace{2pt}\centering{#1}\vspace{2pt}} \\*
\nobreakhline%
!!! Attention !!!& \parbox[top][][c]{\TestProcRightWidth}{\vspace{2pt}#2\vspace{2pt}}\\*
\nobreakhline%
Comments& \parbox[top][#3][t]{\TestProcRightWidth}{\vspace{2pt}}\\*
\hline \hline%
}
%
% \tpNote[1] - autonumbered notes for this test case
%
% #1 is the text of the note
%
\newcommand{\tpNote}[1]{\stepcounter{TestProcItemCounter}~N-\ThisTestProcNum-\theTestProcItemCounter & \parbox[top][][c]{\TestProcRightWidth}{\vspace{2pt}#1\vspace{2pt}} \\*
\hline%
}%
%
% the actual \TestProcedure[6] command definition
%
\newcommand{\TestProcedureBase}[6]{%
\singlespace
% variables to avoid confusion between the #1, #2, and #3 args to \TestProcedure[6], which is also needed in \tpRqmt[1], \tpStep[2], and \tpNote[1]

\renewcommand{\thetestprocedure}{#1}
\renewcommand{\ThisTestProcNum}{#1}% store away the test description number for use in the macro
\renewcommand{\ThisTestProcName}{#2}% store away the test description name for use in the macros
\renewcommand{\ThisTestProcLabel}{#3}% store away the test description label for use in the macros

\renewcommand{\thetestprocstep}{1}
%
% may need to run LaTeX three (3) times to get column widths correct, per LongTable documentation !
%
\begin{longtable}[c]{|c|l|}% base table has two columns <item number & item text>
%
\nobreakhline \nobreakhline%
\multicolumn{2}{|c|}{\parbox[top][12pt][c]{\TestProcWidth}{{\centering \bf Test Procedure \ThisTestProcNum \numberedtestprocedure{\ThisTestProcNum}{\ThisTestProcName}\label{\ThisTestProcLabel}}}}\\*%
\multicolumn{2}{|c|}{\parbox[top][12pt][c]{\TestProcWidth}{\centering \bf \ThisTestProcName}}\\*%
\nobreakhline \hline%
\endfirsthead%
%
\hline \nobreakhline%
\multicolumn{2}{|c|}{\parbox[top][12pt][c]{\TestProcWidth}{\centering \bf Test Procedure \ThisTestProcNum~-- continued from previous page}}\\*%
\nobreakhline\hline%
\endhead%
%
\hline \nobreakhline%
\multicolumn{2}{|r|}{\parbox[top][12pt][c]{\TestProcWidth}{\centering \bf Test Procedure \ThisTestProcNum~-- continues on the next page}}\\*%
\nobreakhline \hline%
\endfoot%
%
\hline \nobreakhline%
\multicolumn{2}{|c|}{\parbox[top][12pt][c]{\TestProcWidth}{\centering \bf End of Test Procedure \ThisTestProcNum}}\\*%
\nobreakhline \hline%
\endlastfoot%
%
% Requirements Validated in this Procedure
%
\hline \nobreakhline%
\multicolumn{2}{|c|}{\parbox[top][12pt][c]{\TestProcWidth}{\centering \bf Requirements Validated by Test Procedure \ThisTestProcNum\setcounter{TestProcItemCounter}{0}}}\\*%
\nobreakhline \nobreakhline%le%
%
#4% can't use {#4} here, need just #4 so it expands, per tex.stackexchange.com answer
%
%%%%%
%
% Notes About This Procedure
%
\hline \nobreakhline%
\multicolumn{2}{|c|}{\parbox[top][12pt][c]{\TestProcWidth}{\centering \bf Notes About Test Procedure \ThisTestProcNum\setcounter{TestProcItemCounter}{0}}}\\*%
\nobreakhline \nobreakhline%
%
#5% can't use {#5} here, need just #5 so it expands, per tex.stackexchange.com answer
%
%%%%%
%
% Steps Performed in this Procedure
%
\hline \nobreakhline%%
\multicolumn{2}{|c|}{\parbox[top][12pt][c]{\TestProcWidth}{\centering \bf Steps to be Performed for Test Procedure \ThisTestProcNum\setcounter{TestProcItemCounter}{0}}}\\*%
\nobreakhline \nobreakhline%
#6% can't use {#6} here, need just #6 so it expands, per tex.stackexchange.com answer

\end{longtable}%
}%

\newcommand{\TestProcedure}[6]{%
	\iftoggle{toggleTestProcStartOnNewPage}{\newpage}{}% force all test procs to start a new page for cleanliness in the document, iff not overridden.
	\TestProcedureBase{#1}{#2}{#3}{#4}{#5}{#6}
}% end of macro \TestProcedure[6]


%%%%%%%%%%%%%%%%%%%%%%%%%%%%%%%%%%%%%%%%%%%%%%%%%%%%%%%
% end of Test Procedure Macro, and auxillary commands %
%%%%%%%%%%%%%%%%%%%%%%%%%%%%%%%%%%%%%%%%%%%%%%%%%%%%%%%