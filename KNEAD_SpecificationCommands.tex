%%%%% Manually numbered requirement
%%%%% table for requirements

\newboolean{boKNEADshowSpecNotes}
\setboolean{boKNEADshowSpecNotes}{true}

\newcommand{\CBVERSION}{UNDEFINED}

\newcommand{\ChangedText}[2]{\ifthenelse{\equal{\CBVERSION}{#1}}{\cbstart #2 \cbend}{#2}}

\newcounter{NRQMT}
\newlistof[NRQMT]{rqtspecification}{aux_spec}{\listspecificationsname}

\newcommand{\numberedrqtspecification}[2]{%
\refstepcounter{rqtspecification}
\par\noindent\textbf{Specification {#1} {#2}}
\addcontentsline{aux_spec}{rqtspecification}{\protect\numberline{#1}#2}\par}

\newcommand{\OneRqmtThreshold}{({\tt T}) - Threshold}
\newcommand{\OneRqmtObjective}{({\tt O}) - Objective}
\newcommand{\OneRqmtInactive} {({\tt I}) - Inactive}
\newcommand{\OneRqmtDeleted}  {({\tt D}) - Deleted}


\newlength{\RqmtMiniPageWidth}
\if@dobindprt % set margins to 1.5" on left (leave room for binding) and 1.0" right
\setlength{\RqmtMiniPageWidth}{4.7in}
\else%!dobindprt so set margins to 1.0" on left and right sides
\setlength{\RqmtMiniPageWidth}{5.2in}
\fi%dobindprt


%%%%
%%%%
%%%% Simple table for listing single requirements with all applicable stuff
%%%%
%%%% #1 is requirement number
%%%% #2 is requirement name
%%%% #3 is requirement label (expected to be of form rqt:XXX)
%%%% #4 is text of the single requirement

%%%% #5 is version that defines this item
%%%% #6 is status (must be a list of \items of form [phase] active, inactive, etc.},#7 is version that defines this item
%%%% #8 is acceptance, #9 is version that defines this item
%%%% #10 is traceability, #11 is version that defines this item
%%%% #12 is notes {must be a list of \items}, #13 is version that defines this item
%%%%
%%%%

\newcommand{\ONERQMTV}[9]{
\renewcommand{\therqtspecification}{#1}
\begin{table}[H]%\usepackage{booktabs} for \toprule, \midrule, \bottomrule
\ifthenelse{\equal{\CBVERSION}{#9}}{\cbstart}{}
\begin{tabular}{@{}r|l@{}}
\toprule
\toprule            
%SPECIFICATION name for list of specifications    
    \multicolumn{2}{@{}c@{}}{%this c only centers the minipage
	   \begin{minipage}{\textwidth}% to allow for it to wrap
	    \centering% to make contents of minipage to be centered
	     \numberedrqtspecification{#1}{#2}
	      \label{#3}
	   \end{minipage}} \\
\midrule            
% TEXT is the formal requirement specification text
			{\bf Text}  & 
			\begin{minipage}{\RqmtMiniPageWidth}
			   {#4} % e.g. The system interface 1 connector 1 as style 38999-XYZ labeled as J1.  
			\end{minipage} \\
\midrule
% STATUS, these are the requirement status by phases
			{\bf Status} &
			\begin{minipage}{\RqmtMiniPageWidth} %%% minipage to allow for multiple specifics in one cell
			 \begin{description}[topsep=0pt,itemsep=-1ex,partopsep=1ex,parsep=1ex]
			    {#5} % MUST BE a list of decription items -- of form \item [Phase] status				 
			 \end{description}  
			\end{minipage} \\			
\midrule
% ACCEPTANCE method (inspection, demonstration, test, analysis, or other		 
			{\bf Acceptance}   & 
			\begin{minipage}{\RqmtMiniPageWidth} %%% minipage to allow for multiple lines of verification			
			   {#6} %e.g. This specification shall be verified by inspection. 
			        % could include other test info here but the formatting is left to user to provide
			\end{minipage} \\
\midrule 
% TRACEABILITY provides a reference to a higher level specification document, if applicable		
			{\bf Traceability} & 
			\begin{minipage}{\RqmtMiniPageWidth} %%% minipage to allow for multiple lines of traceability
			 \begin{description}[topsep=0pt,itemsep=-1ex,partopsep=1ex,parsep=1ex]
			    {#7} % MUST BE a list of decription items -- of form \item [Document] requirement number				 
			 \end{description}						
			%\end{minipage}\\				
%\midrule 
%print out the notes if we're supposed to
% NOTES	are NOT part of the specification, just a place for comments about the specification
\ifthenelse{\boolean{boKNEADshowSpecNotes}}{%	
\end{minipage}\\% end Traceability minipage in the ifthenelse to keep \midrule after the break			
\midrule% seems to need to be right after the break, thus the placement of end{minipage}\\ in the ifthenelse					
			{\bf Notes}        & 
			\begin{minipage}{\RqmtMiniPageWidth} %%% minipage to allow for multiple lines of notes
			 \begin{enumerate}[topsep=0pt,itemsep=-1ex,partopsep=1ex,parsep=1ex]
			    {#8} % MUST BE a list of enumerated items -- of form \item Note 1...
				       % even if there is just one, e.g. \item No notes for this specification.		 
			 \end{enumerate}			    
			\end{minipage}\\	
\bottomrule% end table with double bottomrule right after the final end{minipage} and break
\bottomrule
}%end ifthen clause
{%else just end the table by ending traceability minipage, break, and double bottomrules
\end{minipage}\\				
\bottomrule 
\bottomrule}%end ifthenelse		
\end{tabular}
\ifthenelse{\equal{\CBVERSION}{#9}}{\index{All Changes This Version}\cbend}{}							
\end{table}
} %end of ONERQMTV[9]


%https://tex.stackexchange.com/questions/2132/how-to-define-a-command-that-takes-more-than-9-arguments
\newcommand{\MULTIRQMTV}[1]%looks like [10], #1 is saved and #2...#10 --> #1...#9 in MULTIRQMTVbase[9]
{
\def\SpecNumber{#1}
\MULTIRQMTVbase
}

\newcommand{\MULTIRQMTVbase}[9]{
\renewcommand{\therqtspecification}{\SpecNumber}
\begin{table}[H]%\usepackage{booktabs} for \toprule, \midrule, \bottomrule
\ifthenelse{\equal{\CBVERSION}{#9}}{\cbstart}{}
\begin{tabular}{@{}r|l@{}}
\toprule
\toprule            
%SPECIFICATION name for list of specifications    
    \multicolumn{2}{@{}c@{}}{%this c only centers the minipage
	   \begin{minipage}{\textwidth}% to allow for it to wrap
	    \centering% to make contents of minipage to be centered
	     \numberedrqtspecification{\SpecNumber}{#1}
	      \label{#2}
	   \end{minipage}} \\
\midrule            
% SYNOPSIS is just a quick summary, not the formal rqt specification
			{\bf Text}  & 
			\begin{minipage}{\RqmtMiniPageWidth}
				{#3} %The system interface 1 connector 1 as style 38999-XYZ labeled as J1.  
			\end{minipage} \\
\midrule
% SPECIFICS, these are the formal specification details
			{\bf Specifics} &
			\begin{minipage}{\RqmtMiniPageWidth} %%% minipage to allow for multiple specifics in one cell
			 \begin{enumerate}[topsep=0pt,itemsep=-1ex,partopsep=1ex,parsep=1ex]
				{#4} 
%%%%%			 		\item The connector for interface 1 shall be labled as J1. 
%%%%%				    \item The connector for interface 1 shall be MFG P/N 38999-XYZ.  
%%%%%					\item The connector for interface 1 shall have finish type \TBD.
%%%%%					\item The connector for interface 1 shall have finish color \TBD.
%%%%%					\item The connector for interface 1 shall be keyed as \TBD.							 
			 \end{enumerate}  
			\end{minipage} \\
\midrule
%STATUS  			 
			{\bf Status}   & 
			\begin{minipage}{\RqmtMiniPageWidth} %%% minipage to allow for multiple lines of verification	
				\begin{description}[topsep=0pt,itemsep=-1ex,partopsep=1ex,parsep=1ex]
			   {#5} % MUST BE a list of decription items -- of form \item [Phase] status
				\end{description}
			\end{minipage} \\
\midrule
% ACCEPTANCE method (inspection, demonstration, test, analysis, or other
			{\bf Acceptance}       & 
			\begin{minipage}{\RqmtMiniPageWidth} %%% minipage to allow for multiple lines of status			
			   {#6} %Active 
			\end{minipage} \\	
\midrule
% TRACEABILITY provides a reference to a higher level specification document, if applicable		
			{\bf Traceability} & 
			\begin{minipage}{\RqmtMiniPageWidth} %%% minipage to allow for multiple lines of traceability	
					\begin{description}[topsep=0pt,itemsep=-1ex,partopsep=1ex,parsep=1ex]
			   {#7} %This specification is derived from \ldots.
					\end{description}
			%\end{minipage}\\
%\midrule
%print out the notes if we're supposed to
% NOTES	are NOT part of the specification, just a place for comments about the specification
\ifthenelse{\boolean{boKNEADshowSpecNotes}}{%
\end{minipage}\\% end Traceability minipage in the ifthenelse to keep \midrule after the break			
\midrule% seems to need to be right after the break, thus the placement of end{minipage}\\ in the ifthenelse						
			{\bf Notes}        & 
			\begin{minipage}{\RqmtMiniPageWidth} %%% minipage to allow for multiple lines of notes		
				\begin{enumerate}[topsep=0pt,itemsep=-1ex,partopsep=1ex,parsep=1ex]
				{#8} % MUST BE a list of enumerated items -- of form \item Note 1...
				     % even if there is just one, e.g. \item No notes for this specification.		 
				\end{enumerate}
			\end{minipage} \\	
\bottomrule% end table with double bottomrule right after the final end{minipage} and break
\bottomrule
}%end ifthen clause
{%else just end the table by ending traceability minipage, break, and double bottomrules
\end{minipage}\\				
\bottomrule 
\bottomrule}%end ifthenelse			
\end{tabular}
\ifthenelse{\equal{\CBVERSION}{#9}}{\index{All Changes This Version}\cbend}{}							
\end{table}
}%end of MULTIRQMTVbase[9]




%aliases to find KPP and KSA requirements for the B-spec
\newcommand{\ONERQMTVKPP}{\ONERQMTV}
\newcommand{\ONERQMTVKSA}{\ONERQMTV}
\newcommand{\MULTIRQMTVKPP}{\MULTIRQMTV}
\newcommand{\MULTIRQMTVKSA}{\MULTIRQMTV}
% old, deprecated ones
\newcommand{\ONERQMTKPP}{\ONERQMT}
\newcommand{\ONERQMTKSA}{\ONERQMT}

%%%%%%%%%%%%%%%%%%%%%%%%%%%%%%%%%%%%%%%%%%%%%%%%%%%%%%%%%%%%%%%%%%%%%%%%%%%%%%%%%%%%%%%%%%%%%%%%%%%
%%%%%%%%%%%%%%%%%%%%%%%%%%%%%%%%%%%%%%%%%%%%%%%%%%%%%%%%%%%%%%%%%%%%%%%%%%%%%%%%%%%%%%%%%%%%%%%%%%%
%% Deprecated stuff below here, shouldn't be used, migrate to ones above
%%%%%%%%%%%%%%%%%%%%%%%%%%%%%%%%%%%%%%%%%%%%%%%%%%%%%%%%%%%%%%%%%%%%%%%%%%%%%%%%%%%%%%%%%%%%%%%%%%%
%%%%%%%%%%%%%%%%%%%%%%%%%%%%%%%%%%%%%%%%%%%%%%%%%%%%%%%%%%%%%%%%%%%%%%%%%%%%%%%%%%%%%%%%%%%%%%%%%%%


%%\newcounter{NRQMT}
%%\newlistof[NRQMT]{rqtspecification}{aux_spec}{\listspecificationsname}
%%%%
%%%%
%%%% Simple table for listing single requirements with all applicable stuff
%%%%
%%%% #1 is requirement number
%%%% #2 is requirement name
%%%% #3 is requirement label (expected to be of form rqt:XXX)
%%%% #4 is text of the single requirement
%%%% #5 is status (must be a list of \items of form [phase] active, inactive, etc.}
%%%% #6 is acceptance
%%%% #7 is traceability
%%%% #8 is notes {must be a list of \items}
%%%%
%%%%
\newcommand{\ONERQMT}[8]{
\renewcommand{\therqtspecification}{#1}
\begin{table}[H]%\usepackage{booktabs} for \toprule, \midrule, \bottomrule
\begin{tabular}{@{}r|l@{}}
\toprule
\toprule            
%SPECIFICATION name for list of specifications    
    \multicolumn{2}{@{}c@{}}{%this c only centers the minipage
	   \begin{minipage}{\textwidth}% to allow for it to wrap
	    \centering% to make contents of minipage to be centered
	     \numberedrqtspecification{#1}{#2}
	      \label{#3}
	   \end{minipage}} \\
\midrule            
% TEXT is the formal requirement specification text
			{\bf Text}  & 
			\begin{minipage}{\RqmtMiniPageWidth}
			   {#4} % e.g. The system interface 1 connector 1 as style 38999-XYZ labeled as J1.  
			\end{minipage} \\
\midrule
% STATUS, these are the requirement status by phases
			{\bf Status} &
			\begin{minipage}{\RqmtMiniPageWidth} %%% minipage to allow for multiple specifics in one cell
			 \begin{my_description}
			    {#5} % MUST BE a list of decription items -- of form \item [Phase] status				 
			 \end{my_description}  
			\end{minipage} \\			
\midrule
% ACCEPTANCE method (inspection, demonstration, test, analysis, or other		 
			{\bf Acceptance}   & 
			\begin{minipage}{\RqmtMiniPageWidth} %%% minipage to allow for multiple lines of verification			
			   {#6} %e.g. This specification shall be verified by inspection. 
			        % could include other test info here but the formatting is left to user to provide
			\end{minipage} \\
\midrule 
% TRACEABILITY provides a reference to a higher level specification document, if applicable		
			{\bf Traceability} & 
			\begin{minipage}{\RqmtMiniPageWidth} %%% minipage to allow for multiple lines of traceability
			 \begin{my_description}
			    {#7} % MUST BE a list of decription items -- of form \item [Document] requirement number				 
			 \end{my_description}						
			\end{minipage}\\				
\if@showreqnotes
\midrule 
% NOTES	are NOT part of the specification, just a place for comments about the specification
			{\bf Notes}        & 
			\begin{minipage}{\RqmtMiniPageWidth} %%% minipage to allow for multiple lines of notes
			 \begin{my_enumerate}
			    {#8} % MUST BE a list of items
				     % even if there is just one -- of form \item blah blah			 
			 \end{my_enumerate}			    
			\end{minipage} \\	
\fi%showreqnotes
\bottomrule 								
\bottomrule 								
\end{tabular}
\end{table}
} %end of ONERQMT[8]




















\newcommand{\NRQMTT}[9]{
\renewcommand{\therqtspecification}{#1}
\begin{table}[H]%\usepackage{booktabs} for \toprule, \midrule, \bottomrule


% longtable stuff, still trying to get it to work 20141212WLC
%\begin{longtable}[H]%\usepackage{booktabs} for \toprule, \midrule, \bottomrule
%\hline \hline
%\endhead
%\hline \hline
%\endfirsthead
%\hline \multicolumn{1}{|r|}{{Continued on next page}} \\ \hline
%\endfoot
%\hline \hline
%\endlastfoot

\begin{tabular}{@{}r|l@{}}
\toprule
\toprule            
%SPECIFICATION name for list of specifications    
    \multicolumn{2}{@{}c@{}}{%this c only centers the minipage
	   \begin{minipage}{\textwidth}% to allow for it to wrap
	    \centering% to make contents of minipage to be centered
	     \numberedrqtspecification{#1}{#2}
	      \label{#3}
	   \end{minipage}} \\
\midrule            
% SYNOPSIS is just a quick summary, not the formal rqt specification
			{\bf Text}  & 
			\begin{minipage}{\RqmtMiniPageWidth}
				{#4} %The system interface 1 connector 1 as style 38999-XYZ labeled as J1.  
			\end{minipage} \\
\midrule
% SPECIFICS, these are the formal specification details
			{\bf Specifics} &
			\begin{minipage}{\RqmtMiniPageWidth} %%% minipage to allow for multiple specifics in one cell
			 \begin{my_enumerate}
				{#5} 
%%%%%			 		\item The connector for interface 1 shall be labled as J1. 
%%%%%				    \item The connector for interface 1 shall be MFG P/N 38999-XYZ.  
%%%%%					\item The connector for interface 1 shall have finish type \TBD.
%%%%%					\item The connector for interface 1 shall have finish color \TBD.
%%%%%					\item The connector for interface 1 shall be keyed as \TBD.							 
			 \end{my_enumerate}  
			\end{minipage} \\			
\midrule
% ACCEPTANCE method (inspection, demonstration, test, analysis, or other
			{\bf Acceptance}       & 
			\begin{minipage}{\RqmtMiniPageWidth} %%% minipage to allow for multiple lines of status			
			   {#6} %Active 
			\end{minipage} \\	
\midrule
%%% with the switch to marking each line of the specification, we no longer need to show the overall status
%%% but we leave this option for older documents, just need to add this to options until \RQMTxxx calls are fixed
%STATUS  			 
			{\bf Status}   & 
			\begin{minipage}{\RqmtMiniPageWidth} %%% minipage to allow for multiple lines of verification	
				\begin{my_description}
			   {#7} % MUST BE a list of decription items -- of form \item [Phase] status
				\end{my_description}
			\end{minipage} \\
\midrule 
% TRACEABILITY provides a reference to a higher level specification document, if applicable		
			{\bf Traceability} & 
			\begin{minipage}{\RqmtMiniPageWidth} %%% minipage to allow for multiple lines of traceability	
					\begin{my_description}
			   {#8} %This specification is derived from \ldots.
					\end{my_description}
			\end{minipage}\\				
\if@showreqnotes
\midrule 
% NOTES	are NOT part of the specification, just a place for comments about the specification
	
			{\bf Notes}        & 
			\begin{minipage}{\RqmtMiniPageWidth} %%% minipage to allow for multiple lines of notes		
				\begin{my_enumerate}
				{#9} %There are no notes for this specification.
				\end{my_enumerate}
			\end{minipage} \\	
\fi%showreqnotes
\bottomrule 								
\bottomrule 								
\end{tabular}
\end{table}
%\end{longtable}
}%end of NRQMT[9]


