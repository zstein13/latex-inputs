%%% SVN stuff
\svnidlong
{$HeadURL: https://svn.riouxsvn.com/kneadlatxinputs/ExampleArtifactFolders/7%20-%20SVD/SVD_Chapter_03.tex $}
{$LastChangedDate: 2023-12-26 12:10:35 -0500 (Tue, 26 Dec 2023) $}
{$LastChangedRevision: 42 $}
{$LastChangedBy: KneadProject $}
\svnid{$Id: SVD_Chapter_03.tex 42 2023-12-26 17:10:35Z KneadProject $}


\chapter{Version Description}
\label{loc:VersionDescription}
% \DIDINFO{SVD-3.0.0 :: This chapter provides an overview of the system release.
See reference~\cite{ref__SVD_DID} for more information.}

This chapter will provide an inventory of all used software and hardware, provide change information for the \ThisSystem, and provide installation instructions for required software.

\section{Inventory of Materials}
\label{loc:InventoryOfMaterials}
% \DIDINFO{SVD-3.1.0 :: This section shall list by identifying numbers,
titles, abbreviations, dates, version numbers, and release numbers, as applicable, all physical media (for example, listings, tapes, disks) and associated documentation that make up the software version being released. 
It shall include applicable security and privacy considerations for these items, safeguards for handling them, such as concerns for static and magnetic fields,
and instructions and restrictions regarding duplication and license provisions.}

The materials used for the \ThisSystem are:

\begin{itemize}
    \item Raspberry Pi Pico W
    \begin{itemize}
        \item \href{https://datasheets.raspberrypi.com/picow/pico-w-datasheet.pdf}{Raspberry Pi Pico W Datasheet}
    \end{itemize}
\end{itemize}

The Raspberry Pi Pico W is a RP2040-based microcontroller board with wireless capabilities.

\section{Inventory of Software}
\label{loc:InventoryOfSoftware}
% \DIDINFO{SVD-3.2.0 :: This section shall list by identifying numbers, titles, abbreviations, dates, version numbers, and release numbers, as applicable, all computer files that make up the software version being released. 
Any applicable security and privacy considerations shall be included.}

The software used for the \ThisSystem are:

\begin{itemize}
    \item pico-sdk: Release SDK 1.5.1 June 13, 2023
    \begin{itemize}
        \item \href{https://github.com/raspberrypi/pico-sdk}{pico-sdk Github repository}
    \end{itemize}
    \item pico-examples: Release Tag sdk-1.5.1 June 11, 2023
    \begin{itemize}
        \item \href{https://github.com/raspberrypi/pico-examples}{pico-examples Github repository}
    \end{itemize}
\end{itemize}

Both the pico-sdk and pico-examples repositories are provided by the Raspberry Pi Ltd organization. 
The pico-sdk and pico-examples repositories work on any Raspberry Pi Pico board.

The pico-sdk SDK is a C/C++ SDK designed for the Raspberry Pi Pico family of microcontroller boards. The documentation for pico-sdk can be found in the SDK documentation \cite{ref__RP_Pico_C_SDK}.
The pico-examples repository contains sample code and projects that are provided by the Raspberry Pi Ltd organization. The sample projects use Release 1.5.1 of the pico-sdk. 

% \section{Changes From Prior Version}
% \label{loc:ChangesFromPriorVersion}
% \DIDINFO{SVD-3.3.0 :: This section shall contain a list of all changes incorporated
into the software version since the previous version. If change classes have been used, such as the Class I/Class II changes in MlL-STD-973, the changes shall be separated into these classes.
This paragraph shall identify, as applicable, the problem reports, change proposals, and change notices associated with each change and the effects, if any, of each change on system operation and on interfaces with other hardware and software. 
This paragraph does not apply to the initial system version.}

% This section is \TBD. (And is not applicable to the first release.)


\section{Adaptation Data}
\label{loc:AdaptationData}
% \DIDINFO{SVD-3.4.0 :: This paragraph shall identify or reference all unique-to-site data contained in the software version. 
For software versions after the first, this paragraph shall describe changes made to the adaptation data.}

There is currently no unique-to-site data for the \ThisSystem. Updates will be made to this section when applicable.


\section{Related (Third Party) Documents}
\label{loc:RelatedDocuments}
% \DIDINFO{SVD-3.5.0 :: This paragraph shall list by identifying numbers, titles,
abbreviations, dates, version numbers, and release numbers, as applicable, all documents pertinent to the software version being released but not included in the release.
These artifacts will still be listed in the reference section, and cited here and other places as appropriate; this section just provides a place to further describe these reference documents.}

The following is a list of third party documents associated with the software used for the \ThisSystem:

\begin{itemize}
    \item \cite{ref__RP_Getting_Started} Raspberry Pi Pico Getting Started Guide
    \item \cite{ref__RP_Pico_C_SDK} Raspberry Pi Pico C/C++ SDK Official Documentation
\end{itemize}


\section{Installation Instructions}
\label{loc:InstallationInstructions}
% \DIDINFO{SVD-3.6.0 :: This paragraph shall provide or reference the following
information, as applicable:
a. Instructions for installing the software version; 
b. Identification of other changes that have to be installed for this version to be used,including site-unique adaptation data not included in the software version; 
c. Security, privacy, or safety precautions relevant to the installation; 
d. Procedures for determining whether the version has been installed properly; 
e. A point of contact to be consulted if there are problems or questions with the
installation.}

This section will provide an overview for how to install the pico-sdk and pico-examples repositories.

\subsection{Installing pico-sdk}
For detailed steps on how to install the pico-sdk, please refer to the \cite{ref__RP_Getting_Started} Raspberry Pi Pico Getting Started Guide. 
The pico-sdk can be found in the github repo: \url{github.com/raspberrypi/pico-sdk}
The following will list the necessary steps to install the pico-sdk:

\begin{enumerate}
    \item Install CMake (at least version 3.13), and GCC cross compiler
    \item Clone the Raspberry Pi Pico SDK repository using git clone
    \item Set the ENV variable PICO\textunderscore SDK\textunderscore PATH to the SDK location in your environment.
    \item Setup a CMakeLists.txt file
\end{enumerate}

These steps will install and configure the pico-sdk repository onto your system.

\subsection{Installing pico-examples}
For detailed steps on how to install the pico-examples, please refer to the \cite{ref__RP_Getting_Started} Raspberry Pi Pico Getting Started Guide. 
pico-examples can be found in the github repo: \url{github.com/raspberrypi/pico-examples}
The following will list the necessary steps to install the pico-examples:

\begin{enumerate}
    \item Clone the Raspberry Pi Pico Examples repository using git clone
\end{enumerate}

\subsection{Quick Pico Setup via Install Script}
The Raspberry Pi organization provides a setup shell script to configure your system with the pico-sdk and pico-examples script. 
The Getting Started With Pico \cite{ref__RP_Getting_Started} guide provides more information about the setup script.
The following are the necessary steps to run the install script:

\begin{enumerate}
    \item Get the script via terminal command: wget \url{https://raw.githubusercontent.com/raspberrypi/pico-setup/master/pico_setup.sh}
    \item Make the script executable with: chmod +x pico\textunderscore setup.sh
    \item Run the script with: ./pico\textunderscore setup.sh
\end{enumerate}

\section{Possible Problems and Known Errors}
\label{loc:PossibleProblemsAndKnownErrors}
% \DIDINFO{SVD-3.7.0 :: This paragraph shall identify any possible problems or known errors with the software version at the time of release, any steps being taken to resolve the problems or errors, and instructions (either directly or by reference) for recognizing, avoiding, correcting, or otherwise handling each one. 
The information presented shall be appropriate to the intended recipient of the SVD (for example, a user agency may need advice on avoiding errors, a support agency on correcting them).}

There are no possible problems or known errors at this time. This section will be updated if and when applicable.
