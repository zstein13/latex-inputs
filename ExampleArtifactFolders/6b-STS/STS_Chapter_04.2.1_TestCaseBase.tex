%%% SVN stuff
\svnidlong
{$HeadURL: https://svn.riouxsvn.com/kneadlatxinputs/ExampleArtifactFolders/6b-STS/STS_Chapter_04.x.y_TestCaseBase.tex $}
{$LastChangedDate: 2024-03-14 23:32:07 -0400 (Thu, 14 Mar 2024) $}
{$LastChangedRevision: 101 $}
{$LastChangedBy: KneadProject $}
\svnid{$Id: STS_Chapter_04.x.y_TestCaseBase.tex 101 2024-03-15 03:32:07Z KneadProject $}

\subsection{\StsTestCaseID}
\label{loc:Test\StsTestCaseID}
% \DIDINFO{STD-4.x.y.0 :: This paragraph shall identify a test by project-unique identifier, shall provide a brief description, and shall be divided into the following sub-paragraphs.
When the information required duplicates information previously specified for another test, that information may be referenced rather than repeated.}

Test \StsTestSpecID{Waterproof Test} case \StsTestCaseID{Waterproof Rating} is to validate that the \ThisSystem is passes a IP65 waterproof rating.

\subsubsection{Requirements Addressed}
\label{loc:TestCaseRequirementsAddressed\StsTestSpecID\StsTestCaseID}
% \input{DIDINFO_Snippets/STS/STS_4.x.y.1_DIDINFO.tex}

This requirements validated by test \StsTestSpecID{Waterproof Test} case \StsTestCaseID{Waterproof Rating} are listed in \S~\ref{loc:TestCaseProcedure\StsTestSpecID\StsTestCaseID}.

\subsubsection{Prerequisite Conditions}
\label{loc:TestCasePrerequisiteConditions\StsTestSpecID\StsTestCaseID}
% \DIDINFO{STD-4.x.y.2 :: This paragraph shall identify any prerequisite conditions that must be established prior to performing the test case. 
The following considerations shall be discussed, as applicable: 
(a) hardware and software configuration, 
(b) Flags, initial breakpoints, pointers, control parameters, or initial data to be set/reset prior to test commencement,
(c) Preset hardware conditions or electrical states necessary to run the test case,
(d) Initial conditions to be used in making timing measurements,
(e) Conditioning of the simulated environment, and
(f) Other special conditions peculiar to the test case.}

This prerequisite condition for test \StsTestSpecID{Waterproof Test} case \StsTestCaseID{Waterproof Rating} is a designed external casing for \ThisSystem.

\subsubsection{Inputs}
\label{loc:TestCaseInputs\StsTestSpecID\StsTestCaseID}
% \input{DIDINFO_Snippets/STS/STS_4.x.y.3_DIDINFO.tex}

This inputs for test \StsTestSpecID{Waterproof Test} case \StsTestCaseID{Waterproof Rating} are listed in \S~\ref{loc:TestCaseProcedure\StsTestSpecID\StsTestCaseID}.

\subsubsection{Expected Outputs}
\label{loc:TestCaseExpectedOutputs\StsTestSpecID\StsTestCaseID}
% \input{DIDINFO_Snippets/STS/STS_4.x.y.4_DIDINFO.tex}

This expected outputs for test \StsTestSpecID{Waterproof Test} case \StsTestCaseID{Waterproof Rating} are listed in \S~\ref{loc:TestCaseProcedure\StsTestSpecID\StsTestCaseID}.

\subsubsection{Evaluation Criteria}
\label{loc:TestCaseEvaluationCriteria\StsTestSpecID\StsTestCaseID}
% \DIDINFO{STD-4.x.y.5 :: This paragraph shall identify the criteria to be used for evaluating the intermediate and final results of the test case. 
For each test result, the following information shall be provided, as applicable:
(a) The range or accuracy over which an output can vary and still be acceptable,
(b) Minimum number of combinations or alternatives of input and output conditions that constitute an acceptable test result,
(c) Maximum/minimum allowable test duration, in terms of time or number of events,
(d) Maximum number of interrupts, halts, or other system breaks that may occur,
(e) Allowable severity of processing errors,
(f) Conditions under which the result is inconclusive and re-testing is to be performed,
(g) Conditions under which the outputs are to be interpreted as indicating irregularities in input test data, in the test database/data files, or in test procedures,
(h) Allowable indications of the control, status, and results of the test and the readiness for the next test case (may be output of auxiliary test software), and
(i) Additional criteria not mentioned above.}

This evaluation criteria for test \StsTestSpecID{Waterproof Test} case \StsTestCaseID{Waterproof Rating} are listed in \S~\ref{loc:TestCaseProcedure\StsTestSpecID\StsTestCaseID}.

\subsubsection{Assumptions and Constraints}
\label{loc:TestCaseAssumptions\StsTestSpecID\StsTestCaseID}
% \DIDINFO{STD-4.x.y.6 :: This paragraph shall define the test procedure for the test case. 
The test procedure shall be defined as a series of individually numbered steps listed sequentially in the order in which the steps are to be performed. 
For convenience in document maintenance, the test procedures may be included as an appendix and referenced in this paragraph. 
The appropriate level of detail in each test procedure depends on the type of software being tested. See the DID for more information.}

This assumptions and constraints for test \StsTestSpecID{Waterproof Test} case \StsTestCaseID{Waterproof Rating} are:

\begin{itemize}
    \item A system with an external casing capable of meeting the IP65 waterproof rating
    \item A facility capable of testing and providing waterproof ratings
\end{itemize}

\subsubsection{Procedure}
\label{loc:TestCaseProcedure\StsTestSpecID\StsTestCaseID}
% \DIDINFO{STD-4.x.y.7 :: This paragraph shall identify any assumptions made and constraints or limitations imposed in the description of the test case due to system or test conditions, such as limitations on timing, interfaces, equipment, personnel, and database/data
files. 
If waivers or exceptions to specified limits and parameters are approved, they shall be identified and this paragraph shall address their effects and impacts upon the test case.}

This procedure for test \StsTestSpecID{Waterproof Test} case \StsTestCaseID{Waterproof Rating} is to hire VTEC Laboratories to conduct accelerated weathering tests to specifically test the IP65 rating for the \ThisSystem.

See step~\ref{loc:Step2} for how to reference specific steps.

\TestProcedure%[6] arguments, denoted %N%-NAME
%%%%%%%%%%%%%%%%%%%%%%%%%%%%%%%%%%%%%%%%%
%arg-1 is test procedure number, normally made from section#.X
{%1%-PROCNUM
\getcurrentref{subsubsection}.1
}%1%-PROCNUM
%%%%%%%%%%%%%%%%%%%%%%%%%%%%%%%%%%%%%%%%%
%arg-2 is test procedure name
{%2%-PROCNUM
Waterproofing
}%2%-PROCNUM
%%%%%%%%%%%%%%%%%%%%%%%%%%%%%%%%%%%%%%%%%
%arg-3 is test procedure label, for use in references to the test procedure
{loc:TestProc1}
%
%arg-4 is list of requirements validated in this test procedure label
{%4%-RQMTS
\tpRqmt{Assemble external casing for \ThisSys}
}%4%-RQMTS
%%%%%%%%%%%%%%%%%%%%%%%%%%%%%%%%%%%%%%%%%
%arg-5 is list of notes for this test procedure label
{%5%-NOTES
\tpNote{External Casing must protect environmental sensors}
}%5%-NOTES
%%%%%%%%%%%%%%%%%%%%%%%%%%%%%%%%%%%%%%%%%
%arg-6 is list of steps for this test procedure label
% provided as a list of special commands tpStepXYZ[2], tpStep[3], or tpStepLabeled[4]
{%6%-STEPS
%
%RECORD PRE-TEST INFORMATION
\tpStepINFO{START RECORDING OF PRE-TEST INFORMATION}
%
\tpStep%{Action}{Expected Result}{Space to record results}
{Record Date and Time at Start of Test}
{Date and Time at Start of Test are recorded.}
{40pt}
%
\tpStep%{Action}{Expected Result}{Space to record results}
{Record Name of Test Engineer(s) and Agency}
{Name of Test Engineer(s) and their Agency are recorded.}
{40pt}
%
\tpStep%{Action}{Expected Result}{Space to record results}
{Record Name of Witness(es) and Agency}
{Name of Witness(es) and their Agency are recorded.}
{40pt}
%
\tpStep%{Action}{Expected Result}{Space to record results}
{Record configuration information or name of file that contains such information.}
{Configuration information or name of file that contains such information is recorded.}
{100pt}
%
\tpStepBULLSEYE{END RECORDING OF PRE-TEST INFORMATION}
%from $TEXINPUTS; make a local copy and adjust as appropriate.
%
% use these as section delineators as needed; pick image that best fits the need
% \tpStepCLOCK{CLOCK TEXT -- good to show that some time must elapse} 
% \tpStepHAND{HAND TEXT -- use, with text, when tester needs to pause and double check things} 
% \tpStepINFO{Configure connection script and connect Raspberry Pi Pico W to WiFi network.}
% \tpStepKEY{KEY TEXT -- good to make a key point, or if something needs to be locked/unlocked}
% \tpStepMAGNIFY{MAGNIFY TEXT -- note info that magnifies what is happening} 
% \tpStepPLAYARROW{PLAYARROW TEXT -- denote a starting point, such as when test stations change.
% This is just more text to see what happens when there are 3 or 4 lines of text w.r.t. centering of icon.
% These text blocks should be short, but, could be long, so this checks to see what happens with 5 or 6 lines of text.}
\tpStepBANG{Prepare external casing for waterproof testing}% 
% \tpStepSHOCK{SHOCK TEXT -- denote a HAZARD}
% \tpStepRADIATION{RADIATION TEXT -- denote an EXTREME HAZARD} 
% \tpStepQUESTION{QUESTION TEXT -- ensure a question is answered} 
% \tpStepBULLSEYE{BULLSEYE TEXT -- denote the end of a mini-sequence}
%
\tpStep%{Action}{Expected Result}{Space to record results}
{Undergo advanced weathering test at VTEC Laboratories}
{\ThisSystem withstands weathering test and meets IP65 waterproof rating}
{20pt}
{loc:Step1}
%
% \tpStepLabeled%{Action}{Expected Result}{Space to record results}{label for use in \reference{}}
% {Configure python test script with Raspberry Pi Pico W IP Address.}
% {Test script configured}
% {20pt}
% {loc:Step2}
%
% This one shows how to insert an image in the test step;
% most like to show expected results, but can be used as test step to show the step if it makes sense
% \tpStepLabeled%{Action}{Expected Result}{Space to record results}{label for use in \reference{}}
% {Run python test script and record the number of packets sent and received.}
% {Packets are sent to and received by the Raspberry Pi Pico W.}
% {20pt}
% {loc:Step3}
%
% This one shows how to insert an image in the test step;
% most like to show expected results, but can be used as test step to show the step if it makes sense
% \tpStepLabeled%{Action}{Expected Result}{Space to record results}{label for use in \reference{}}
% {\tpStepFigure{images/KNEAD_UnderConstruction_100dpi_6.5inchesWide.eps}{1.0in}}
% {Labeled step 4 works!}
% {20pt}
% {loc:Step4}
%
%RECORD POST-TEST INFORMATION
\tpStepINFO{START RECORDING OF POST-TEST INFORMATION}
%
\tpStep%{Action}{Expected Result}{Space to record results}
{Record Date and Time at End of Test}
{Date and Time at End of Test are recorded.}
{40pt}
%
\tpStep%{Action}{Expected Result}{Space to record results}
{Record Signature of Test Engineer(s)}
{Signature of Test Engineer(s) is recorded.}
{40pt}
%
\tpStep%{Action}{Expected Result}{Space to record results}
{Record Signature of Witness(es)}
{Signature of Witness(es) are recorded.}
{40pt}
%
\tpStep%{Action}{Expected Result}{Space to record results}
{Record any pertinent comment about the test procedure, results, and/or environment.}
{Any pertinent comment about the test procedure, results, and/or environment is recorded.}
{200pt}
%
\tpStepBULLSEYE{END RECORDING OF POST-TEST INFORMATION}
%from $TEXINPUTS; make a local copy and adjust as appropriate.
%
}%6%-STEPS
%%%%%%%%%%%%%%%%%%%%%%%%%%%%%%%%%%%%%%%%%


