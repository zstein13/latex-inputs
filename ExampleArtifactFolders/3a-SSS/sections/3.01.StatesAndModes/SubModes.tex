
A summary of the sub-modes is provided in Table~\ref{tab:Sub-Modes}.
This table also provides a list of the mode in which each sub-mode is valid.
See the formal specifications, if applicable, in the following sections for formal statement of the sub-mode requirements, and accompanying notes that provide further clarification on the meanings of the states.
\begin{table}[htbp]
	\begin{center}
		\begin{tabular}{|p{1.0in}|p{4.0in}|p{1.0in}|}
			\hline
			\hline
			     \multicolumn{3}{|c|}{{\bf SUB-MODES}}\\
			\hline
{\bf Name} & {\bf Summary}	& {\bf Valid Sub-States}\\
			\hline
			\hline
Temperature Monitoring & \ThisSys will monitor temperature and activate the irrigation system based on a temperature threshold & Active Monitoring\\ \hline
Soil Moisture Level Monitoring & \ThisSys will monitor soil moisture and activate the irrigation system based on a moisture threshold & Active Monitoring\\ \hline

			\hline
			\hline
			\end{tabular}
				\caption{Summary of Sub-Modes for \ThisSystem}
				\label{tab:Sub-Modes}
		\end{center}
\end{table}


%%%
%%% include subsections and/or files with requirements for the sub-states as needed
%%%

%%% \ONERQMTV[9] is macro for consistently formatted requirements
\ONERQMTV
% #1 is requirement Number
{\RqtNumberBase.1}
% #2 is Title
{Temperature Monitoring}
% #3 is requirement label (expected to be of form rqt:XXX)
{rqt:SubModeA}
% #4 is Text of the specification
{The system shall monitor environmental temperature and activate the irrigation system for a set duration if the temperature is above a set temperature threshold.}
% #5 is Status (S in {(T), (O), (I), (D)} listed by phase as \item [] S)
{
	\item [All Phases] Threshold
}
% #6 is Acceptance
{This requirement shall be verified by demonstration.}
% #7 is Traceability
{
	\item [N/A] This requirement is a base requirement.
}
% #8 is Notes; listed as enumeration \item ...
{
	\item The Temperature Monitoring submode generalizes the case where the system is actively monitoring temperature.
	\item When temperature crosses a set threshold, the irrigation system will activate for a set duration. 
	\item After the set time duration has passed, the irrigation system will deactivate.
}
% #9 is changebar version
{P1}
%%%%% end \ONERQMTV[9] macro

%%% \ONERQMTV[9] is macro for consistently formatted requirements
\ONERQMTV
% #1 is requirement Number
{\RqtNumberBase.2}
% #2 is Title
{Soil Moisture Level Monitoring}
% #3 is requirement label (expected to be of form rqt:XXX)
{rqt:SubModeB}
% #4 is Text of the specification
{The system shall monitor soil moisture levels and activate the irrigation system when moisture levels drop below a certain threshold.}
% #5 is Status (S in {(T), (O), (I), (D)} listed by phase as \item [] S)
{
	\item [All Phases] Threshold
}
% #6 is Acceptance
{This requirement shall be verified by demonstration.}
% #7 is Traceability
{
	\item [N/A] This requirement is a base requirement.
}
% #8 is Notes; listed as enumeration \item ...
{
	\item The Soil Moisture Level Monitoring submode generalizes the case where the system is actively monitoring soil moisture levels.
	\item When soil moisture levels drop below a certain threshold, the irrigation system will activate until soil moisture levels are above a certain threshold.
	\item After soil moisture levels are above a certain threshold, the irrigation system will be deactivated.
}
% #9 is changebar version
{P1}
%%%%% end \ONERQMTV[9] macro

