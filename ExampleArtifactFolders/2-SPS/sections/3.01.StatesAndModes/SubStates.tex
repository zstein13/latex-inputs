A summary of the sub-states is provided in Table~\ref{tab:SubStates}.
This table also provides a list of the states in which each sub-state is valid.
See the formal specifications, if applicable, in the following sections for formal statement of the sub-state requirements, and accompanying notes that provide further clarification on the meanings of the states.
\begin{table}[h]
	\begin{center}
		\begin{tabular}{|p{1.0in}|p{4.0in}|p{1.0in}|}
			\hline
			\hline
			\multicolumn{3}{|c|}{{\bf SUB-STATES}} \\
			\hline
				{\bf Sub-State Name} & {\bf Summary} & {\bf Valid States} \\ 
			\hline
			\hline
Sub State A & summary & State 1 \\ \hline
Sub State B & summary & State 2 \\ \hline
Sub State C & summary & State 3 \\ 
			\hline
			\hline
		\end{tabular}
		\caption{Summary of Sub-States for \ThisSystem}
		\label{tab:SubStates}
	\end{center}
\end{table}

%%%
%%% include subsections and/or files with requirements for the sub-states as needed
%%%

%%% \ONERQMTV[9] is macro for consistently formatted requirements
\ONERQMTV
% #1 is requirement Number
{\RqtNumberBase.1}
% #2 is Title
{SubState A}
% #3 is requirement label (expected to be of form rqt:XXX)
{rqt:SubStateA}
% #4 is Text of the specification
{The system shall provide the SubState-A substate.}
% #5 is Status (S in {(T), (O), (I), (D)} listed by phase as \item [] S)
{
	\item [Phase 1] Threshold
}
% #6 is Acceptance
{This requirement shall be verified by demonstration.}
% #7 is Traceability
{
	\item [N/A] This requirement is a base requirement.
}
% #8 is Notes; listed as enumeration \item ...
{
	\item The SubState-A substate generalizes the case where the system is \TBD.
}
% #9 is changebar version
{P1}
%%%%% end \ONERQMTV[9] macro

%%% \ONERQMTV[9] is macro for consistently formatted requirements
\ONERQMTV
% #1 is requirement Number
{\RqtNumberBase.2}
% #2 is Title
{SubState B}
% #3 is requirement label (expected to be of form rqt:XXX)
{rqt:SubStateB}
% #4 is Text of the specification
{The system shall provide the SubState-B substate.}
% #5 is Status (S in {(T), (O), (I), (D)} listed by phase as \item [] S)
{
	\item [Phase 1] Threshold
}
% #6 is Acceptance
{This requirement shall be verified by demonstration.}
% #7 is Traceability
{
	\item [N/A] This requirement is a base requirement.
}
% #8 is Notes; listed as enumeration \item ...
{
	\item The SubState-B substate generalizes the case where the system is \TBD.
}
% #9 is changebar version
{P1}
%%%%% end \ONERQMTV[9] macro


%%% \ONERQMTV[9] is macro for consistently formatted requirements
\ONERQMTV
% #1 is requirement Number
{\RqtNumberBase.3}
% #2 is Title
{SubState C}
% #3 is requirement label (expected to be of form rqt:XXX)
{rqt:SubStateC}
% #4 is Text of the specification
{The system shall provide the SubState-C substate.}
% #5 is Status (S in {(T), (O), (I), (D)} listed by phase as \item [] S)
{
	\item [Phase 1] Threshold
}
% #6 is Acceptance
{This requirement shall be verified by demonstration.}
% #7 is Traceability
{
	\item [N/A] This requirement is a base requirement.
}
% #8 is Notes; listed as enumeration \item ...
{
	\item The SubState-C substate generalizes the case where the system is \TBD.
}
% #9 is changebar version
{P1}
%%%%% end \ONERQMTV[9] macro
