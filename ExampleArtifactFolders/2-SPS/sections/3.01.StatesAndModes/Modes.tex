
A summary of the modes is provided in Table~\ref{tab:Modes}.
This table also provides a list of the sub-states in which each mode is valid.
See the formal specifications, if applicable, in the following sections for formal statement of the mode requirements, and accompanying notes that provide further clarification on the meanings of the states.
\begin{table}[htbp]
	\begin{center}
		\begin{tabular}{|p{1.0in}|p{4.0in}|p{1.0in}|}
			\hline
			\hline
			     \multicolumn{3}{|c|}{{\bf MODES}}\\
			\hline
{\bf Name} & {\bf Summary}	& {\bf Valid Sub-States}\\
			\hline
			\hline
Mode 1 & Mode 1 summary & Sub-State A\\ \hline
Mode 2 & Mode 2 summary & Sub-State B\\ \hline
Mode 3 & Mode 3 summary & Sub-State C\\ 
			\hline
			\hline
			\end{tabular}
				\caption{Summary of Modes for \ThisSystem}
				\label{tab:Modes}
		\end{center}
\end{table}

%%%
%%% include subsections and/or files with requirements for the sub-states as needed
%%%

%%% \ONERQMTV[9] is macro for consistently formatted requirements
\ONERQMTV
% #1 is requirement Number
{\RqtNumberBase.1}
% #2 is Title
{Mode One}
% #3 is requirement label (expected to be of form rqt:XXX)
{rqt:ModeOne}
% #4 is Text of the specification
{The system shall provide the Mode-1 mode.}
% #5 is Status (S in {(T), (O), (I), (D)} listed by phase as \item [] S)
{
	\item [Phase 1] Threshold
}
% #6 is Acceptance
{This requirement shall be verified by demonstration.}
% #7 is Traceability
{
	\item [N/A] This requirement is a base requirement.
}
% #8 is Notes; listed as enumeration \item ...
{
	\item The Mode-1 mode generalizes the case where the system is \TBD.
}
% #9 is changebar version
{P1}
%%%%% end \ONERQMTV[9] macro


%%% \ONERQMTV[9] is macro for consistently formatted requirements
\ONERQMTV
% #1 is requirement Number
{\RqtNumberBase.2}
% #2 is Title
{Mode Two}
% #3 is requirement label (expected to be of form rqt:XXX)
{rqt:ModeTwo}
% #4 is Text of the specification
{The system shall provide the Mode-2 mode.}
% #5 is Status (S in {(T), (O), (I), (D)} listed by phase as \item [] S)
{
	\item [Phase 1] Threshold
}
% #6 is Acceptance
{This requirement shall be verified by demonstration.}
% #7 is Traceability
{
	\item [N/A] This requirement is a base requirement.
}
% #8 is Notes; listed as enumeration \item ...
{
	\item The Mode-2 mode generalizes the case where the system is \TBD.
}
% #9 is changebar version
{P1}
%%%%% end \ONERQMTV[9] macro


%%% \ONERQMTV[9] is macro for consistently formatted requirements
\ONERQMTV
% #1 is requirement Number
{\RqtNumberBase.3}
% #2 is Title
{Mode Three}
% #3 is requirement label (expected to be of form rqt:XXX)
{rqt:ModeThree}
% #4 is Text of the specification
{The system shall provide the Mode-3 mode.}
% #5 is Status (S in {(T), (O), (I), (D)} listed by phase as \item [] S)
{
	\item [Phase 1] Threshold
}
% #6 is Acceptance
{This requirement shall be verified by demonstration.}
% #7 is Traceability
{
	\item [N/A] This requirement is a base requirement.
}
% #8 is Notes; listed as enumeration \item ...
{
	\item The Mode-3 mode generalizes the case where the system is \TBD.
}
% #9 is changebar version
{P1}
%%%%% end \ONERQMTV[9] macro