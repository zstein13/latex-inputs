\KNEADSECTIONNEWPAGE
\section{Adaptation Requirements}
\label{lab:sec_Adaptation}
\RequirementNumberAM{section}{3.6}
% \DIDINFO{SPS/SSS-3.6.0 :: This section shall specify the requirements, if any, concerning installation-dependent data that the system is required to provide (such as site dependent latitude and longitude or site-dependent state tax codes) and operational parameters that the system is required to use that may vary according to operational needs (such as parameters indicating operation-dependent targeting constants or data recording).}



This section lists the adaptation requirements for the system. The \ThisSys capability is segmented into the following specification groups:

\begin{description}
	\item [Temperature Conversions] describes the adaptation requirements pertaining to temperature measurements, \S~\ref{loc:ADAPTATION_TEMP}.
\end{description}

\KNEADSUBSECTIONNEWPAGE
\subsection{Temperature Conversions}
\label{loc:ADAPTATION_TEMP}
\RequirementNumberAM{subsection}{3.6.1}

%%% \ONERQMTVKSA[9] is macro for consistently formatted requirements
\ONERQMTV
% #1 is requirement Number
{\RqtNumberBase.1}
% #2 is Title
{Temperature Conversions}
% #3 is requirement label (expected to be of form rqt:XXX)
{rqt:Temperature_Conversions}
% #4 is Text of the specification
{Temperature data measured must be capable of being converted to \degree F and \degree C}
% #5 is Status (S in {(T), (O), (I), (D)} listed by phase as \item [] S)
{
	\item [All Phases] Threshold
}
% #6 is Acceptance
{This requirement shall be verified by demonstration.}
% #7 is Traceability
{
\item [N/A] This is a base requirement.
}
% #8 is Notes; listed as enumeration \item ...
{
	\item Temperature can be converted between Farenheit and Celcius.
}
% #9 is changebar version
{P1}
%%%%% end \ONERQMTV[9] macro