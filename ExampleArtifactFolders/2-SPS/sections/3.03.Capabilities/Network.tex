The network requirements for \ThisSystem are listed below.


%%% \ONERQMTV[9] is macro for consistently formatted requirements
\ONERQMTVKPP
% #1 is requirement Number
{\RqtNumberBase.1}
% #2 is Title
{Network Types}
% #3 is requirement label (expected to be of form rqt:XXX)
{rqt:NetworkTypes}
% #4 is Text of the specification
{All \ThisSys variants shall be capable of \TBD network types.}
% #5 is Status (S in {(T), (O), (I), (D)} listed by phase as \item [] S)
{
	\item [Phase 1] Threshold
}
% #6 is Acceptance
{This requirement shall be verified by demonstration.}
% #7 is Traceability
{
	\item [N/A] This requirement is a base requirement.
}
% #8 is Notes; listed as enumeration \item ...
{
  \item Add as many of these as necessary.  Split into files/folders, e.g., NetworkTypes.tex, NetworkInputs.tex, and NetworkOutputs.tex, etc. as needed. Just use the RequirementNumberAM and RqtNumberBase commands to keep numbers correct if subsubsections are added.
}
% #9 is changebar version
{P1}
%%%%% end \ONERQMTV[9] macro


%%% \ONERQMTV[9] is macro for consistently formatted requirements
\ONERQMTVKPP
% #1 is requirement Number
{\RqtNumberBase.2}
% #2 is Title
{Network Inputs}
% #3 is requirement label (expected to be of form rqt:XXX)
{rqt:NetworkInputs}
% #4 is Text of the specification
{All \ThisSys variants shall be capable of \TBD network inputs.}
% #5 is Status (S in {(T), (O), (I), (D)} listed by phase as \item [] S)
{
	\item [Phase 1] Threshold
}
% #6 is Acceptance
{This requirement shall be verified by demonstration.}
% #7 is Traceability
{
	\item [N/A] This requirement is a base requirement.
}
% #8 is Notes; listed as enumeration \item ...
{
  \item Add as many of these as necessary.  Split into files/folders, e.g., NetworkTypes.tex, NetworkInputs.tex, and NetworkOutputs.tex, etc. as needed. Just use the RequirementNumberAM and RqtNumberBase commands to keep numbers correct if subsubsections are added.
}
% #9 is changebar version
{P1}
%%%%% end \ONERQMTV[9] macro

%%% \ONERQMTV[9] is macro for consistently formatted requirements
\ONERQMTVKPP
% #1 is requirement Number
{\RqtNumberBase.3}
% #2 is Title
{Network Outputs}
% #3 is requirement label (expected to be of form rqt:XXX)
{rqt:NetworkOutputs}
% #4 is Text of the specification
{All \ThisSys variants shall be capable of \TBD network outputs.}
% #5 is Status (S in {(T), (O), (I), (D)} listed by phase as \item [] S)
{
	\item [Phase 1] Threshold
}
% #6 is Acceptance
{This requirement shall be verified by demonstration.}
% #7 is Traceability
{
	\item [N/A] This requirement is a base requirement.
}
% #8 is Notes; listed as enumeration \item ...
{
  \item Add as many of these as necessary.  Split into files/folders, e.g., NetworkTypes.tex, NetworkInputs.tex, and NetworkOutputs.tex, etc. as needed. Just use the RequirementNumberAM and RqtNumberBase commands to keep numbers correct if subsubsections are added.
}
% #9 is changebar version
{P1}
%%%%% end \ONERQMTV[9] macro
