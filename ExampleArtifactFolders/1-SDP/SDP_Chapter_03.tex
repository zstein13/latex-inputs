%%% SVN stuff
\svnidlong
{$HeadURL: https://svn.riouxsvn.com/kneadlatxinputs/ExampleArtifactFolders/1%20-%20SDP/SDP_Chapter_03.tex $}
{$LastChangedDate: 2023-12-27 21:52:08 -0500 (Wed, 27 Dec 2023) $}
{$LastChangedRevision: 44 $}
{$LastChangedBy: KneadProject $}
\svnid{$Id: SDP_Chapter_03.tex 44 2023-12-28 02:52:08Z KneadProject $}


\chapter{Required Work Overview}
\label{loc:RequiredWorkOverview}
\DIDINFO{SDP-3.0.0 :: This chapter provides an overview of the required work for development of \ThisSystem.
See reference~\cite{ref__SDP_DID} for more information.
It shall include, as applicable, an overview of:
a. Requirements and constraints on the system to be developed;
b. Requirements and constraints on project documentation;
c. Position of the project in the system life cycle;
d. The selected program/acquisition strategy or any requirements or constraints on it;
e. Requirements and constraints on project schedules and resources;
f. Other requirements and constraints, such as on project security, privacy, methods, standards, interdependencies in hardware and software development, etc.}

This chapter is \TBD.

\section{Program Status}
\label{loc:RWO_ProgramStatus}
\input{DIDINFO_Snippets/SDP/SDP_3.1.0_DIDINFO.tex}

This section is \TBD.


\section{SDLC Situation}
\label{loc:RWO_SDLC}
\input{DIDINFO_Snippets/SDP/SDP_3.1.0_DIDINFO.tex}

This section is \TBD.


\section{Requirement Plans}
\label{loc:RWO_RequirementPlans}
\DIDINFO{SDP-3.2.0 :: This chapter provides an overview of the required work for development of \ThisSystem.
See reference~\cite{ref__SDP_DID} for more information.
}

This section is \TBD.


\section{Documentation Plans}
\label{loc:RWO_DocumentationPlans}
\input{DIDINFO_Snippets/SDP/SDP_3.3.0_DIDINFO.tex}

This section is \TBD.

The following documents are listed here just to test reference generation.
A ``real'' \SDP would reference these as applicable for the project.
\begin{itemize}
	\item \citeExProjOCD is the \OCD, which outlines the project overall so, generally, it is created first.
	\item \citeExProjSDP is this document.
	\item \citeExProjSPS is the \SPS, which should come from the customer or end user, but often is generated by the developer with customer approval.
	\item \citeExProjSSS is the \SSS that is the developer's design specification to meet the \SPS requirements.
	\item \citeExProjSUM is the \SUM that acts somewhat like part of the \SSS since it illustrates the \UI design part of the \SSS, but in a separate artifact that also can be used as a standalone users' manual.
	\item \citeExProjHRS is a \HRS, which often is not used for smaller projects but can have multiple instances for large projects to more fully detail hardware design.	
	\item \citeExProjSRS is a \SRS, which often is not used for smaller projects but can have multiple instances for large projects to more fully detail software or firmware design.
	\item \citeExProjIRS is the \IRS, which often is not use but may be needed, even if \HRS or \SRS artifacts are not, to fully document detailed interfaces such as Application Programming Interfaces (\APIs) or other detailed mechanical or electrical interfaces.
	\item \citeExProjSSDD is the \SSDD that provides a road map to the design and other design details needed to understand the hardware and software design.
	\item \citeExProjSTP is the \STP that highlights the planning for system testing.
	\item \citeExProjSTS is the \STS, which is sometimes called a test procedure. There could be multiple of these based on the overall project size.
	\item \citeExProjSTR is an \STR that documents the results of a given test. Multiple instances are expected based on the test plan. And, there could be multiple versions of a given test plan to document repeated occurrences of a given test specification/procedure.
	\item \citeExProjSVD is an \SVD that documents a given release of a system. Multiple versions of these ``release notes'' are expected, with one \SVD issued for each system release cycle.

\end{itemize}



\section{Schedule and Resource Constraints}
\label{loc:RWO_ScheduleAndResourceConstraints}
\input{DIDINFO_Snippets/SDP/SDP_3.4.0_DIDINFO.tex}

This section is \TBD.


\section{Other Constraints}
\label{loc:RWO_OtherConstraints}
\input{DIDINFO_Snippets/SDP/SDP_3.5.0_DIDINFO.tex}

This section is \TBD.