%%% SVN stuff
\svnidlong
{$HeadURL: https://svn.riouxsvn.com/kneadlatxinputs/ExampleArtifactFolders/0%20-%20SDP/SDP_Chapter_04.tex $}
{$LastChangedDate: 2023-12-24 20:45:24 -0500 (Sun, 24 Dec 2023) $}
{$LastChangedRevision: 27 $}
{$LastChangedBy: KneadProject $}
\svnid{$Id: SDP_Chapter_04.tex 27 2023-12-25 01:45:24Z KneadProject $}


\chapter{System Development Plans}
\label{loc:PlansForSystemDevelopment}
% \DIDINFO{SDP-4.0.0 :: This section shall be divided into paragraphs as needed to
establish the context for the planning described in later sections. 
}

This chapter will provide a brief overview of the required hardware and firmware for /ThisSystem. 

\section{Hardware Development Plans}
\label{loc:SDP_HardwareDevelopmentPlans}
% \input{DIDINFO_Snippets/SDP/SDP_4.1.0_DIDINFO.tex}

There will be no hardware development for this project. All development will take place on a Raspberry Pi Pico W development kit.

\section{Firmware Development Plans}
\label{loc:SDP_FirmwareDevelopmentPlans}
% \input{DIDINFO_Snippets/SDP/SDP_4.2.0_DIDINFO.tex}

The firmware development plan will follow the development assignment outline for the course ENPM818I.
The development assignments are as follows:

\begin{itemize}
    \item Dev Assignment 1
    \begin{itemize}
        \item Development kit and tools are operational
    \end{itemize}

    \item Dev Assignment 2
    \begin{itemize}
        \item Debugger and Emulator tools for the devlopment kit are operational
    \end{itemize}
    \item Dev Assignment 3
    \begin{itemize}
        \item Demonstrate necessary IO Stream and Interrupt Handler for the project.
    \end{itemize}
    \item Dev Assignment 4
    \begin{itemize}
        \item Demonstrate security measures necessary for the project.
    \end{itemize}
    \item Dev Assignment 5
    \begin{itemize}
        \item Implement networking capabilities for the project.
    \end{itemize}
\end{itemize}

\section{Software Development Plans}
\label{loc:SDP_SoftwareDevelopmentPlans}
% \input{DIDINFO_Snippets/SDP/SDP_4.3.0_DIDINFO.tex}

\subsection{Raspberry Pi SDK Installation}
The required steps to locally setup the Raspberry Pi SDK for the Raspberry Pi Pico W are as follows:

\begin{itemize}
    \item SDK Documents with references
\end{itemize}

The software development plan for this project is still \TBD. This section will be updated in the future.


\section{Integration Plans}
\label{loc:SDP_IntegrationPlans}
% \input{DIDINFO_Snippets/SDP/SDP_4.4.0_DIDINFO.tex}

The integration plans for this project are still \TBD. This section will be updated in the future.


\section{Testing Plans}
\label{loc:SDP_TestingPlans}
% \input{DIDINFO_Snippets/SDP/SDP_4.5.0_DIDINFO.tex}

All firmware and software for \ThisSystem will be accompanied with unit tests. 
An open source tool will be selected to audit the code base and there will be a set threshold for code coverage. 
Otherwise, the formal testing plan for \ThisSystem is still \TBD and will be updated in the future.


\section{Other Development Activities}
\label{loc:SDP_OtherDevelopmentActivities}
% \input{DIDINFO_Snippets/SDP/SDP_4.6.0_DIDINFO.tex}

Other development activities will be added to this document in the future. As of now, this section is \TBD.