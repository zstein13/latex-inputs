\DIDINFO{SSDD-4.1.0 :: This section shall
\begin{itemize}
\item Identify the components of the system (HWCIs, CSCIs, and manual operations). Each component shall be assigned a project-unique identifier. Note: a database may be treated as a CSCI or as part of a CSCI.
\item Show the static (such as “consists of’) relationship(s) of the components. Multiple relationships may be presented, depending on the selected design methodology.
\item State the purpose of each component and identify the system requirements and system-wide design decisions allocated to it. (Alternatively, the allocation of requirements may be provided in 5.a).
\item Identify each component’s development status/type, if known (such as new development, existing component to be reused as is, existing design to be reused as is, existing design or component to be reengineered, component to be developed for reuse, component planned for Build N, etc.) For existing design or components, the description shall provide identifying information, such as name, version, documentation references, location, etc.
\item For each computer system or other aggregate of computer hardware resources identified for use in the system, describe its computer hardware resources (such as processors, memory, input/output devices, auxiliary storage, and communications/network equipment). Each description shall, as applicable, identify the configuration items that will use the resource, describe the allocation of resource utilization to each CSCI that will use the resource (for example, 20% of the resource’s capacity allocated to CSCI 1, 30% to CSCI 2), describe the conditions under which utilization will be measured, and describe the characteristics of the
resources.
\item Present a specification tree for the system, that is, a diagram that identifies and shows the relationships among the planned specifications for the system components.
\end{itemize}
}