\DIDINFO{STP-4.1.0 :: This section shall shall be divided into subsections as applicable to present general information applicable to the overall testing to be performed.
The general information covers areas such as:
\begin{description}
\item[Test Levels] such as the levels at which testing will be performed, for example, CSCI level or system level, or stages su as defined in \cite[].
\item[Test Classes] such as the types or classes of tests that will be
performed (for example, timing tests, erroneous input tests, maximum capacity tests)
\item[General Test Conditions] such as conditions that apply to all of the tests or to a group of tests. 
For example: Each test shall include nominal, maximum, and
minimum values;” “each test of type x shall use live data;” “execution size and time shall be measured for each CSCI.” 
Included shall be a statement of the extent of testing to be performed and rationale for the extent selected. The extent of testing shall be expressed as a percentage of some well defined total quantity, such as the number of samples of discrete operating conditions or values, or other sampling approach. 
Also included shall be the approach to be followed for retesting/regressing testing.
\item[Test Progression] such as in case of progressive or cumulative tests, where the planned sequence or progression of tests must be defined.
\item[Data recording, reduction, and analysis] shall identify and describe the data recording, reduction, and analysis procedures to be used during and after the tests identified in this STP. 
These procedures shall include, as applicable, manual, automatic, and semi-automatic techniques for recording test results, manipulating the raw results into a form suitable for evaluation, and retaining the results of data reduction and analysis.
\end{description}}