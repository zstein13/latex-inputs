This document format is based upon the guidance in the \STD{} \DID~\cite{ref__STD_DID} and the \STR{} \DID~\cite{ref__STR_DID}.
The test planning is documented following the guidelines of ISO-12207~\cite{ref__ISO_12207} and MIL-STD-498~\cite{ref__MIL_STD_498} (from which ISO-12207 originated).
This document follows the listed \STS sub-section order.
\begin{description}
	\item[Section 1] provides an overview of the system and this document.
	\item[Section 2] lists general and application-specific reference documents as well as glossary terms and acronyms. 
	\item[Section 3] summarizes the test preparations.
	\item[Section 4] provides the detailed descriptions of the tests to be performed\footnote{This section follows the \DID but places the test procedure details as the last section for each test case.}. 
	\item[Section 5] provides any applicable requirement traceability.
	\item[Section 6] if needed, lists any general notes as may be applicable.
	\item[Appendices] if needed, provide additional information as may be needed.
\end{description}


This document also is structured to serve as the basis for the system test report (\STR).
Each test is supplied with spaces for capturing pertinent hardware, software, and other log information.
Each test is divided into one or more test cases, each with detailed steps, expected results for each step, and a set of easy to read pass/fail markers for each test step.
All tests steps also provide space to fill in the results and to write notes and comments about each test step.
The goal of this style is to generate this \STR by scanning in the resultant \STS with comments using this as the  \STR Appendix-A.
In this manner, a ``written'' record of the testing is generated, thus saving money by not requiring a completely separate recording document for the \STR.

% \vspace{12pt}
\DIDINFO{If this text is visible, the first instance of each section may display a summary of data item description (DID) information shown in this font.
These are displayed in small capital font and are not part of the formal document.}
