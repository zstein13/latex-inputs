The test strategy follows a seven-stage test process, which is defined in the Department of Defense Design Criteria Standard -- Development of Shipboard Industrial Test Procedures~\cite{ref__DOD_STD_2106_NAVY}.
The first three stages occur at the component level and the second three stages occur at the assembly, sub-system, and system level.
The final stage is complete system testing in its deployed configuration.
The test stages are defined as:
\begin{enumerate}
	\item {\bf Material Receipt Inspection :: } to ensure material received from vendor have received no damage during shipping and parameters match expectations, e.g. part numbers match purchase order information and paperwork matches content.  Also an Return Materiel Authorization (\RMA) would be started, if required.
	\item {\bf Bench Functional Test :: } of components expected configuration (e.g. correct memory, licenses, etc.) and operation as expected in a stand-alone manner (e.g. power-up and basic operational testing).
	\item {\bf Installation / Installed Equipment Test :: } of special requirements at the installed component level (e.g. shock, vibration, \EMI, etc. of \LRUs).
	\item {\bf Intra-system Test  :: } of sub-systems basic operation as expected in a stand-alone manner (e.g. \SWaP, limited functionality, etc.).
	\item {\bf Total System Test :: } of the entire system in a test environment that may impart some restrictions on final functionality (e.g. limited communications connections or dummy loads instead of antennas).
	\item {\bf System Special Test :: } of special requirements at the sub-system level (e.g. shock, vibration, \EMI, \EMC, etc. of entire rack(s) of \LRUs).
	\item {\bf Total Integrated System Tests :: } of the entire system in a deployed environment that imparts no restrictions on final functionality.
\end{enumerate}