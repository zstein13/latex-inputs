This section provides the requirements that define the performance required of the \ThisSys.
These specifications are divided into the major capabilities of the system.
Each requirement is listed in the area that provides the specified capability, thus this section provides an immediate mapping of performance requirements to the capability area in which each requirement is met.
Each requirement also includes traceability to high-level requirements that drive the specific capability.
Validation methodology is provided here but verification traceability is provided in the \STS artifacts.


The requirements also are specified in an order that generally allows for all precursor requirements to be stated before they are needed by a successor requirement.
Thus, States and Modes are defined at the onset so that they can be used to regulate when external interfaces and processing steps may occur.
Likewise, external interfaces are described so that the data from the interfaces may be used in, or created by, the ensuing processing.
Once the processing is specified, the internal interface and data requirements are listed, showing how the overall system segments tie together.
The remainder of the sections follow a somewhat similar pattern, but these latter sections contain disparate requirements that are separated and organized in a standard way so the contents can be easily scanned to locate specific requirements based on their type and expected location within the \SPS.

These specifications also include qualifications for both Threshold (must meet) and Objective (want to meet) requirements for \ThisSys.
The reader is cautioned to ensure that the requirement details be understood for the two modifiers.
